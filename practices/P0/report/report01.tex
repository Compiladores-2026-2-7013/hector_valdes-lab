\documentclass[12pt, a4paper]{article}

% --- Paquetes ---
\usepackage[utf8]{inputenc}
\usepackage[T1]{fontenc}
\usepackage[spanish]{babel}
\usepackage{amsmath, amssymb}
\usepackage{enumitem}
\usepackage{geometry}
\usepackage{hyperref}
\usepackage{listings}

\geometry{margin=2.5cm}

% --- Configuración de listings ---
\lstset{
  basicstyle=\ttfamily\small,
  breaklines=true,
  frame=single,
  language=bash
}

% --- Datos ---
\title{Práctica 0 --- Etapas de Compilación \\[0.5em]
\large Compiladores --- Semestre 2026-2}
\author{Hector Valdes}
\date{}

\begin{document}
\maketitle

% ============================================================
\section{Introducción}
% ============================================================

En esta práctica se exploran las etapas del proceso de compilación de un
programa escrito en lenguaje C, desde el preprocesamiento hasta el enlazado.
Se utiliza cada herramienta de forma individual para observar las
transformaciones que sufre el código fuente en cada fase.

% ============================================================
\section{Desarrollo}
% ============================================================

% ------------------------------------------------------------
\subsection{Preprocesamiento (\texttt{cpp})}
% ------------------------------------------------------------

Se ejecutó el siguiente comando para generar el archivo preprocesado:

\begin{lstlisting}
cpp programa.c programa.i
\end{lstlisting}

\begin{enumerate}[label=\alph*)]
  \item ¿Qué ocurre cuando se invoca el comando \texttt{cpp} con esos argumentos?

  \item ¿Qué similitudes encuentra entre los archivos \texttt{programa.c} y
        \texttt{programa.i}?

  \item ¿Qué pasa con las macros y los comentarios del código fuente original
        en \texttt{programa.i}?

  \item Compare el contenido de \texttt{programa.i} con el de \texttt{stdio.h}
        e indique de forma general las similitudes entre ambos archivos.

  \item ¿A qué etapa corresponde este proceso?
\end{enumerate}

% ------------------------------------------------------------
\subsection{Compilación (\texttt{gcc -S})}
% ------------------------------------------------------------

Se ejecutó la siguiente instrucción:

\begin{lstlisting}
gcc -Wall -S programa.i
\end{lstlisting}

\begin{enumerate}[label=\alph*)]
  \item ¿Para qué sirve la opción \texttt{-Wall}?

  \item ¿Qué le indica a \texttt{gcc} la opción \texttt{-S}?

  \item ¿Qué contiene el archivo de salida y cuál es su extensión?

  \item ¿A qué etapa corresponde este comando?
\end{enumerate}

% ------------------------------------------------------------
\subsection{Ensamblado (\texttt{as})}
% ------------------------------------------------------------

Se ejecutó la siguiente instrucción:

\begin{lstlisting}
as programa.s -o programa.o
\end{lstlisting}

\begin{enumerate}[label=\alph*)]
  \item Antes de revisarlo, indique cuál es su hipótesis sobre lo que debe
        contener el archivo con extensión \texttt{.o}.

  \item Diga de forma general qué contiene el archivo \texttt{programa.o} y
        por qué se visualiza de esa manera.

  \item ¿Qué programa se invoca con \texttt{as}?

  \item ¿A qué etapa corresponde la llamada a este programa?
\end{enumerate}

% ------------------------------------------------------------
\subsection{Enlazado (\texttt{ld})}
% ------------------------------------------------------------

Se buscó la ruta de los siguientes archivos en el equipo de trabajo:

\begin{itemize}
  \item \texttt{ld-linux-x86-64.so.2}
  \item \texttt{Scrt1.o} (o bien, \texttt{crt1.o})
  \item \texttt{crti.o}
  \item \texttt{crtbeginS.o}
  \item \texttt{crtendS.o}
  \item \texttt{crtn.o}
\end{itemize}

Se ejecutó el siguiente comando, sustituyendo las rutas encontradas:

\begin{lstlisting}
ld -o ejecutable -dynamic-linker /lib/ld-linux-x86-64.so.2 \
   /usr/lib/Scrt1.o /usr/lib/crti.o programa.o \
   -lc /usr/lib/crtn.o
\end{lstlisting}

\begin{enumerate}[label=\alph*)]
  \item En caso de que el comando \texttt{ld} mande errores, investigue cómo
        enlazar un programa utilizando \texttt{ld} y proponga una posible
        solución para llevar a cabo este proceso con éxito.

  \item Describa el resultado obtenido al ejecutar el comando anterior.
\end{enumerate}

% ------------------------------------------------------------
\subsection{Ejecución}
% ------------------------------------------------------------

Una vez que se enlazó el código máquina relocalizable, se ejecutó el programa
con la siguiente instrucción:

\begin{lstlisting}
./ejecutable
\end{lstlisting}

% ------------------------------------------------------------
\subsection{Modificación de la macro \texttt{\#define PI}}
% ------------------------------------------------------------

Se quitó el comentario de la macro \texttt{\#define PI} en el código fuente
original y se realizó lo siguiente:

\begin{enumerate}[label=\alph*)]
  \item Se generó nuevamente el archivo \texttt{.i} con un nuevo nombre.

  \item ¿Cambia en algo la ejecución final?
\end{enumerate}

% ------------------------------------------------------------
\subsection{Segundo programa con directivas del preprocesador}
% ------------------------------------------------------------

Se escribió un segundo programa en lenguaje C que incluye 4 directivas del
preprocesador distintas entre sí y diferentes de las utilizadas en el primer
programa.\footnote{Puede consultarse la lista de directivas en la documentación
en línea:
\url{https://gcc.gnu.org/onlinedocs/cpp/Index-of-Directives.html}.
También se puede revisar la entrada para el preprocesador con
\texttt{man cpp}.}

\begin{enumerate}[label=\alph*)]
  \item Explique la utilidad general de cada directiva y su función particular
        en el programa.
\end{enumerate}

% ============================================================
\section{Resultados y Conclusiones}
% ============================================================



\end{document}
